% -*- coding: utf-8; -*-
% vim: set fileencoding=utf-8 :
\documentclass[english,submission]{programming}
%% First parameter: the language is 'english'.
%% Second parameter: use 'submission' for initial submission, remove it for camera-ready (see 5.1)

\usepackage[backend=biber]{biblatex}
\addbibresource{example.bib}


%
% Packages and Commands specific to article (see 3)
%
% These ones  are used in the guide, replace with your own.
% 
\usepackage{multicol}
\lstdefinelanguage[programming]{TeX}[AlLaTeX]{TeX}{%
  deletetexcs={title,author,bibliography},%
  deletekeywords={tabular},
  morekeywords={abstract},%
  moretexcs={chapter},%
  moretexcs=[2]{title,author,subtitle,keywords,maketitle,titlerunning,authorinfo,affiliation,authorrunning,paperdetails,acks,email},
  moretexcs=[3]{addbibresource,printbibliography,bibliography},%
}%
\lstset{%
  language={[programming]TeX},%
  keywordstyle=\firamedium,
  stringstyle=\color{RosyBrown},%
  texcsstyle=*{\color{Purple}\mdseries},%
  texcsstyle=*[2]{\color{Blue1}},%
  texcsstyle=*[3]{\color{ForestGreen}},%
  commentstyle={\color{FireBrick}},%
}

\newcommand*{\CTAN}[1]{\href{http://ctan.org/tex-archive/#1}{\nolinkurl{CTAN:#1}}}
%%


%%%%%%%%%%%%%%%%%%
%% These data MUST be filled for your submission. (see 5.3)
\paperdetails{
  %% perspective options are: art, sciencetheoretical, scienceempirical, engineering.
  %% Choose exactly the one that best describes this work. (see 2.1)
  perspective=art,
  %% State one or more areas, separated by a comma. (see 2.2)
  %% Please see list of areas in http://programming-journal.org/cfp/
  %% The list is open-ended, so use other areas if yours is/are not listed.
  area={Social Coding, General-purpose programming},
  %% You may choose the license for your paper (see 3.)
  %% License options include: cc-by (default), cc-by-nc
  % license=cc-by,
}
%%%%%%%%%%%%%%%%%%

%%%%%%%%%%%%%%%%%%
%% These data are provided by the editors. May be left out on submission.
%\paperdetails{
%  submitted=2016-08-10,
%  published=2016-10-11,
%  year=2016,
%  volume=1,
%  issue=1,
%  articlenumber=1,
%}
%%%%%%%%%%%%%%%%%%


\begin{document}

\title{Live probes for free}
%\subtitle{Preparing Articles for Programming}% optional
%\titlerunning{Preparing Articles for Programming} %optional, in case that the title is too long; the running title should fit into the top page column

\author[a]{Tobias Pape}
\authorinfo{is the author of this {LaTeX} class. Contact him at
  \email{tobias.pape@hpi.uni-potsdam.de}.}
\affiliation[a]{Hasso Plattner Institute, University of Potsdam, Germany}
\author{Cristina V. Lopes}
\authorinfo{is associate editor for the first two issues of The Art, Science,
  and Engineering of Programming. Contact her at \email{lopes@ics.uci.edu}.}
\affiliation{University of California, Irvine, USA}
\author[a]{Robert Hirschfeld}
\authorinfo{is chair of the AOSA steering committee. The Art, Science,
  and Engineering of Programming is published by AOSA. Contact Robert at \email{hirschfeld@hpi.uni-potsdam.de}.}

% \authorrunning{T. Pape, C. Lopes, R. Hirschfeld} % Optional, for long author lists

\keywords{programming journal, paper formatting, submission preparation} % please provide 1--5 keywords


%%%%%%%%%%%%%%%%%%%%%%%%%%%%%
% Please go to https://dl.acm.org/ccs/ccs.cfm and generate your Classification
% System [view CCS TeX Code] stanz and copy _all of it_ to this place.
%% From HERE
\begin{CCSXML}
<ccs2012>
<concept>
<concept_id>10002944.10011122.10003459</concept_id>
<concept_desc>General and reference~Computing standards, RFCs and guidelines</concept_desc>
<concept_significance>300</concept_significance>
</concept>
<concept>
<concept_id>10010405.10010476.10010477</concept_id>
<concept_desc>Applied computing~Publishing</concept_desc>
<concept_significance>300</concept_significance>
</concept>
</ccs2012>
\end{CCSXML}

\ccsdesc[300]{General and reference~Computing standards, RFCs and guidelines}
\ccsdesc[500]{Applied computing~Publishing}

% To HERE
%%%%%%%%%%%%%%%%%%%%%%%

\maketitle

% Please always include the abstract.
% The abstract MUST be written according to the directives stated in 
% http://programming-journal.org/submission/
% Failure to adhere to the abstract directives may result in the paper
% being returned to the authors.
\begin{abstract}
  %What is the broad context of the work? What is the importance of the general research area?
  \emph{Context}
  % What problem or question does the paper address? How has this problem or question been addressed by others (if at all)?
  \emph{Inquiry}
  % What was done that unveiled new knowledge?
  \emph{Approach}
  % What new facts were uncovered? If the research was not results oriented, what new capabilities are enabled by the work?
  \emph{Knowledge}
  % What argument, feasibility proof, artifacts, or results and evaluation support this work?
  \emph{Grounding}
  % Why does this work matter?
  \emph{Importance}
 
\end{abstract}
\section{Introduction}
\label{sec:introduction}
\section{Problem Overview}
\label{sec:problem-overview}
\section{Stack Recording}
\label{sec:stack-recording}
\section{Live Probes in Java with JDI}
\label{sec:live-probes-java}
\section{Generalizing Live Probes with Debugger Adapter Protocol}
\label{sec:generalizing-live-probes}
\section{Evaluation}
\label{sec:evaluation}
\section{Related Work}
\label{sec:related-work}
\section{Conclusion}
\label{sec:conclusion}

\printbibliography

\end{document}

% Local Variables:
% TeX-engine: luatex
% End:
